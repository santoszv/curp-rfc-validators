\documentclass[12pt,letterpaper]{article}

\usepackage[spanish,mexico]{babel}
\usepackage{hyperref}
\usepackage{fontspec}
\usepackage{enumitem}
\usepackage{moreverb}

\setmainfont{CMU Serif}

\begin{document}

\title{Validadores para CURP y RFC}
\author{Santos Zatarain Vera <santoszv(at)inftel.com.mx>}

\maketitle

\tableofcontents

\section{Validación de CURP}

Para validar la \emph{Clave Única de Registro de Población} (CURP) se debe utilizar una o varias de
las siguientes anotaciones:

\begin{enumerate}[noitemsep]
\item \texttt{@mx.com.inftel.contraints.curp.CURP}
\item \texttt{@mx.com.inftel.contraints.curp.CURP.PalabraInconveniente}
\item \texttt{@mx.com.inftel.contraints.curp.CURP.DigitoVerificador}
\end{enumerate}

La anotación (1) valida si la cadena de texto contiene un CURP bien formado, la anotación (2) valida que no exista
alguna palabra inconveniente al principio de la cadena de texto y por último, la anotación (3) calcula y valida
el dígito verificador. Todas las anotaciones etán implementadas como \emph{regular constraints} según lo especificado
en \emph{JSR 349 Bean Validation Specification (formerly JSR 303)}.

Todas anotaciones tienen los parámetros comunes para \emph{constraints}:

\begin{description}[noitemsep]
\item[message] La plantilla del mensaje de error
\item[groups] Los grupos de validación
\item[payload] Carga adicional del \emph{constraint}
\end{description}

Los parámetros comunes usan los valores por defecto recomendados según lo especificado en \emph{JSR 349 Bean
Validation Specification (formerly JSR 303)}.

Solo la anotación (1) tiene dos parámetros adicionales a los comunes:

\begin{description}[noitemsep]
\item[isNullValueValid] Indicación para aceptar a \texttt{null} (nulo) como un valor válido, por defecto
    es \texttt{true}
\item[isEmptyValueValid] Indicación para aceptar a \texttt{\textquotedbl\textquotedbl} (cadena vacía) como un valor
    válido, por defecto es \texttt{false}
\end{description}

Cada anotación contiene el sub-elemento \texttt{List} para especificar de forma múltiple el \emph{constraint}, de
acuerdo las recomendaciones del \emph{JSR 349 Bean Validation Specification (formerly JSR 303)}.

\subsection{Ejemplo 1}

Validación simple, solo se comprueba que esté bien formado el CURP.

\begin{listing}{1}
package my.pkg;

import mx.com.inftel.contraints.curp.CURP;

public class JavaBean{

    @CURP
    private String curp;

    public JavaBean(){}

    public String getCurp(){return curp;}

    public void setCurp(String curp){this.curp = curp;}
}
\end{listing}

\subsection{Ejemplo 2}

Validación de CURP, la cadena vacía también es aceptada como válida. Esta forma es usada por aplicaciones
web principalmente, los formularios regularmente son enviados con campos vacios.

\begin{listing}{1}
package my.pkg;

import mx.com.inftel.contraints.curp.CURP;

public class JavaBean{

    @CURP(isEmptyValueValid = true)
    private String curp;

    public JavaBean(){}

    public String getCurp(){return curp;}

    public void setCurp(String curp){this.curp = curp;}
}
\end{listing}

\subsection{Ejemplo 3}

Validación de CURP, con comprobación del dígito verificador.

\begin{listing}{1}
package my.pkg;

import mx.com.inftel.contraints.curp.CURP;
import mx.com.inftel.contraints.curp.CURP.DigitoVerificador;

public class JavaBean{

    @CURP
    @DigitoVerificador
    private String curp;

    public JavaBean(){}

    public String getCurp(){return curp;}

    public void setCurp(String curp){this.curp = curp;}
}
\end{listing}

\section{Validación de RFC}

Para validar el \emph{Registro Federal de Contribuyente} (RFC) se debe utilizar una o varias de
las siguientes anotaciones:

\begin{enumerate}[noitemsep]
\item \texttt{@mx.com.inftel.contraints.rfc.RFC}
\item \texttt{@mx.com.inftel.contraints.rfc.RFC.PalabraInconveniente}
\item \texttt{@mx.com.inftel.contraints.rfc.RFC.DigitoVerificador}
\end{enumerate}

La anotación (1) valida si la cadena de texto contiene un RFC bien formado, la anotación (2) valida que no exista
alguna palabra inconveniente al principio de la cadena de texto y por último, la anotación (3) calcula y valida
el dígito verificador. Todas las anotaciones etán implementadas como \emph{regular constraints} según lo especificado
en \emph{JSR 349 Bean Validation Specification (formerly JSR 303)}.

Todas anotaciones tienen los parámetros comunes para \emph{constraints}:

\begin{description}[noitemsep]
\item[message] La plantilla del mensaje de error
\item[groups] Los grupos de validación
\item[payload] Carga adicional del \emph{constraint}
\end{description}

Los parámetros comunes usan los valores por defecto recomendados según lo especificado en \emph{JSR 349 Bean
Validation Specification (formerly JSR 303)}.

Solo la anotación (1) tiene cuatro parámetros adicionales a los comunes:

\begin{description}[noitemsep]
\item[isNullValueValid] Indicación para aceptar a \texttt{null} (nulo) como un valor válido, por defecto
    es \texttt{true}
\item[isEmptyValueValid] Indicación para aceptar a \texttt{\textquotedbl\textquotedbl} (cadena vacía) como un valor
    válido, por defecto es \texttt{false}
\item[isXAXX010101000ValueValid] Indicación para aceptar a \texttt{\textquotedbl{}XAXX010101000\textquotedbl}
    (RFC para público general) como un valor válido, por defecto es \texttt{false}
\item[isXEXX010101000ValueValid] Indicación para aceptar a \texttt{\textquotedbl{}XEXX010101000\textquotedbl}
    (RFC para extranjero) como un valor válido, por defecto es \texttt{false}
\end{description}

Cada anotación contiene el sub-elemento \texttt{List} para especificar de forma múltiple el \emph{constraint}, de
acuerdo las recomendaciones del \emph{JSR 349 Bean Validation Specification (formerly JSR 303)}.

\subsection{Ejemplo 1}

Validación simple, solo se comprueba que esté bien formado el RFC.

\begin{listing}{1}
package my.pkg;

import mx.com.inftel.contraints.rfc.RFC;

public class JavaBean{

    @RFC
    private String rfc;

    public JavaBean(){}

    public String getRfc(){return rfc;}

    public void setRfc(String rfc){this.rfc = rfc;}
}
\end{listing}

\subsection{Ejemplo 2}

Validación con grupos, para validar también el dígito verificador, es necesario usar el grupo \texttt{ConDigito}.

\begin{listing}{1}
package my.pkg;

import mx.com.inftel.contraints.rfc.RFC;
import mx.com.inftel.contraints.rfc.RFC.DigitoVerificador;

import javax.validation.groups.Default;

public class JavaBean{

    @RFC(groups = {Default.class, ConDigito.class})
    @DigitoVerificador(groups = {ConDigito.class})
    private String rfc;

    public JavaBean(){}

    public String getRfc(){return rfc;}

    public void setRfc(String rfc){this.rfc = rfc;}
}
\end{listing}

\subsection{Ejemplo 3}

Validación de RFC, la cadena vacía también es aceptada como válida, también es válido
\texttt{\textquotedbl{}XAXX010101000\textquotedbl} y \texttt{\textquotedbl{}XEXX010101000\textquotedbl}.

\begin{listing}{1}
package my.pkg;

import mx.com.inftel.contraints.rfc.RFC;
import mx.com.inftel.contraints.rfc.RFC.DigitoVerificador;

public class JavaBean{

    @RFC(isEmptyValueValid = true,
         isXAXX010101000ValueValid = true,
         isXEXX010101000ValueValid = true)
    private String rfc;

    public JavaBean(){}

    public String getRfc(){return rfc;}

    public void setRfc(String rfc){this.rfc = rfc;}
}
\end{listing}

\section{Plantillas de los mensajes de error}

Las plantillas \emph{por defecto} están escritas en español. Para comodidad de los traductores,
también se incluyen plantillas traducidas al inglés.

\subsection{ValidationMessages.properties}

\scriptsize
\begin{listing}{1}
mx.com.inftel.contraints.curp.CURP.message=Este CURP está mal formado
mx.com.inftel.contraints.curp.CURP.PalabraInconveniente.message=Este CURP contiene una palabra inconveniente
mx.com.inftel.contraints.curp.CURP.DigitoVerificador.message=El dígito verificador no es correcto
mx.com.inftel.contraints.rfc.RFC.message=Este RFC no está bien formado
mx.com.inftel.contraints.rfc.RFC.PalabraInconveniente.message=Este RFC contiene una palabra inconveniente
mx.com.inftel.contraints.rfc.RFC.DigitoVerificador.message=El dígito verificador no es correcto
\end{listing}
\normalsize

\subsection{ValidationMessages\textunderscore{}en.properties}

\scriptsize
\begin{listing}{1}
mx.com.inftel.contraints.curp.CURP.message=This CURP is not well-formed
mx.com.inftel.contraints.curp.CURP.PalabraInconveniente.message=This CURP has a drawback word
mx.com.inftel.contraints.curp.CURP.DigitoVerificador.message=The check digit is not correct
mx.com.inftel.contraints.rfc.RFC.message=This RFC is not well-formed
mx.com.inftel.contraints.rfc.RFC.PalabraInconveniente.message=This RFC has a drawback word
mx.com.inftel.contraints.rfc.RFC.DigitoVerificador.message=The check digit is not correct
\end{listing}
\normalsize

\section{Integración con Maven}

Este proyecto está sincronizado en el \emph{Maven Central Repository}, así que,
para integrar las anotaciones a un proyecto \emph{Maven}, solo se debe agregar
la dependencia, como lo muestra el siguiente \emph{snippet}:

\begin{verbatim}
<dependency>
    <artifactId>mx.com.inftel.oss</artifactId>
    <groupId>curp-rfc-validators</groupId>
    <version>3.0-SNAPSHOT</version>
</dependency>
\end{verbatim}

\section{Notas}

Existen casos de RFC registrados ante el SAT con el dígito verificador incorrecto, por lo que
es recomendable deshabilitar el \emph{constraint} de validación del dígito verificador si es necesario.

\section{Contacto y sugerencias}

Es posible contactar directamente escribiendo un correo electrónico a
\emph{Santos Zatarain Vera <santoszv(at)inftel.com.mx>}, se agradece usar
como asunto \emph{Validadores CURP/RFC}.

\section{Licencia}

\begin{verbatim}
Copyright 2016 Santos Zatarain Vera

Licensed under the Apache License, Version 2.0 (the "License");
you may not use this file except in compliance with the License.
You may obtain a copy of the License at

    http://www.apache.org/licenses/LICENSE-2.0

Unless required by applicable law or agreed to in writing, software
distributed under the License is distributed on an "AS IS" BASIS,
WITHOUT WARRANTIES OR CONDITIONS OF ANY KIND, either express or implied.
See the License for the specific language governing permissions and
limitations under the License.
\end{verbatim}

\end{document}